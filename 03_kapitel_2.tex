%----------------------------------------------------------------
% Chapter 2
% https://de.wikipedia.org/wiki/Matrix_(Mathematik)
%----------------------------------------------------------------
\chapter{Wichtige Mathematik Formeln}
\label{cha:wichtige_mathematik_formeln}


%----------------------------------------------------------------
% Section 1
%----------------------------------------------------------------
\section{Matrix}
\label{sec:matrix}
In der Mathematik versteht man unter einer Matrix (Plural Matrizen) eine rechteckige Anordnung (Tabelle) von Elementen (meist mathematischer Objekte, etwa Zahlen). Mit diesen Objekten lässt sich dann in bestimmter Weise rechnen, indem man Matrizen addiert oder miteinander multipliziert. Matrizen können beliebige Dimensionalität besitzen bspw. \ref{eq:matrix}.
\vspace{5mm}
Matrizen sind ein Schlüsselkonzept der linearen Algebra und tauchen in fast allen Gebieten der Mathematik auf. Sie stellen Zusammenhänge, in denen Linearkombinationen eine Rolle spielen, übersichtlich dar und erleichtern damit Rechen- und Gedankenvorgänge.
\vspace{5mm}

\[ \begin{array}{|cccc|} \label{eq:matrix}
a_{11} & a_{12} & \cdots & a_{1n} \\
a_{21} & a_{22} & \cdots & a_{21} \\
\vdots & \vdots & \ddots & \vdots \\
a_{m1} & a_{m2} & \cdots & a_{mn}
\end{array} \]

\begin{displaymath}
\left\{\begin{array}{cccc}
\Gamma_{11} & \Gamma_{12} & \cdots &
\Gamma_{1n}\\
\Gamma_{21} & \Gamma_{22} & \cdots &
\Gamma_{2n}\\
\vdots & \vdots & \ddots &
\vdots\\
\Gamma_{m1} & \Gamma_{m2} & \cdots &
\Gamma_{mn}
\end{array}\right\}
\end{displaymath}


